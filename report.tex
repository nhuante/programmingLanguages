\documentclass{article}

\usepackage{amsthm}
\usepackage{amsfonts}
\usepackage{amsmath}
\usepackage{amssymb}
\usepackage{fullpage}
\usepackage[usenames]{color}
\usepackage{hyperref}
  \hypersetup{
    colorlinks = true,
    urlcolor = blue,       % color of external links using \href
    linkcolor= blue,       % color of internal links 
    citecolor= blue,       % color of links to bibliography
    filecolor= blue,        % color of file links
    }
    
\usepackage{listings}

\definecolor{dkgreen}{rgb}{0,0.6,0}
\definecolor{gray}{rgb}{0.5,0.5,0.5}
\definecolor{mauve}{rgb}{0.58,0,0.82}

\lstset{frame=tb,
  language=haskell,
  aboveskip=3mm,
  belowskip=3mm,
  showstringspaces=false,
  columns=flexible,
  basicstyle={\small\ttfamily},
  numbers=none,
  numberstyle=\tiny\color{gray},
  keywordstyle=\color{blue},
  commentstyle=\color{dkgreen},
  stringstyle=\color{mauve},
  breaklines=true,
  breakatwhitespace=true,
  tabsize=3
}

\theoremstyle{theorem} 
   \newtheorem{theorem}{Theorem}[section]
   \newtheorem{corollary}[theorem]{Corollary}
   \newtheorem{lemma}[theorem]{Lemma}
   \newtheorem{proposition}[theorem]{Proposition}
\theoremstyle{definition}
   \newtheorem{definition}[theorem]{Definition}
   \newtheorem{example}[theorem]{Example}
\theoremstyle{remark}    
  \newtheorem{remark}[theorem]{Remark}


\title{CPSC-354 Report}
\author{Natalie Huante  \\ Chapman University}

\date{\today}

\begin{document}

\maketitle

\begin{abstract}
Short  summary of purpose and content.  
\end{abstract}

\tableofcontents

\section{Introduction}\label{intro}

\section{Homework}\label{homework}

This section contains solutions to homework. 

\subsection{Week 1}

The homework for Week 1 is dedicated to allow myself the opportunity to get familiar with LaTeX as well as review the model of equational reasoning. 
In this lesson, we used the Fibonacci Sequence as an example of this but we will use the function of Greatest Common Divisor for the assignment.
In terms of familiarity with LaTeX, I do have experience through the Algorithm Analysis report from the previous 
semester, however, this serves as a way to remind myself of the language. \\


For context, the definition the GCD function is as follows:
\begin{verbatim}
gcd(a,b): 
Input: Two whole numbers (integers) called a and b, both greater than 0.
(1) if a>b then replace a by a-b and go to (1).
(2) if b>a then replace b by b-a and go to (1).
Output: a
\end{verbatim}

Now, I will write out the full computation of gcd(9,33) below:

\begin{align*}
  gcd(9,33) & = gcd(9,24)\\
            & = gcd(9,15)\\
            & = gcd(9,6)\\
            & = gcd(3,6)\\
            & = gcd(3,3)\\
            & = 3
\end{align*}

As you can see above, the basis of the function is to iteratively decrement a and b until you reach a point at which you can no longer
decrement them by the definition given. 

\subsection{Week 2}

\ldots

\section{Conclusions}\label{conclusions}

(approx 400 words) A critical reflection on the content of the course. Step back from the technical details. How does the course fit into the wider world of software engineering? What did you find most interesting or useful? What improvements would you suggest?

\begin{thebibliography}{99}
\bibitem[ALG]{Alg} \href{https://github.com/alexhkurz/algorithm-analysis-2023}{Algorithm Analysis}, Chapman University, 2023.
\end{thebibliography}

\end{document}